% Options for packages loaded elsewhere
\PassOptionsToPackage{unicode}{hyperref}
\PassOptionsToPackage{hyphens}{url}
%
\documentclass[
]{article}
\usepackage{amsmath,amssymb}
\usepackage{lmodern}
\usepackage{ifxetex,ifluatex}
\ifnum 0\ifxetex 1\fi\ifluatex 1\fi=0 % if pdftex
  \usepackage[T1]{fontenc}
  \usepackage[utf8]{inputenc}
  \usepackage{textcomp} % provide euro and other symbols
\else % if luatex or xetex
  \usepackage{unicode-math}
  \defaultfontfeatures{Scale=MatchLowercase}
  \defaultfontfeatures[\rmfamily]{Ligatures=TeX,Scale=1}
\fi
% Use upquote if available, for straight quotes in verbatim environments
\IfFileExists{upquote.sty}{\usepackage{upquote}}{}
\IfFileExists{microtype.sty}{% use microtype if available
  \usepackage[]{microtype}
  \UseMicrotypeSet[protrusion]{basicmath} % disable protrusion for tt fonts
}{}
\makeatletter
\@ifundefined{KOMAClassName}{% if non-KOMA class
  \IfFileExists{parskip.sty}{%
    \usepackage{parskip}
  }{% else
    \setlength{\parindent}{0pt}
    \setlength{\parskip}{6pt plus 2pt minus 1pt}}
}{% if KOMA class
  \KOMAoptions{parskip=half}}
\makeatother
\usepackage{xcolor}
\IfFileExists{xurl.sty}{\usepackage{xurl}}{} % add URL line breaks if available
\IfFileExists{bookmark.sty}{\usepackage{bookmark}}{\usepackage{hyperref}}
\hypersetup{
  pdftitle={Chapter 4-(1)},
  hidelinks,
  pdfcreator={LaTeX via pandoc}}
\urlstyle{same} % disable monospaced font for URLs
\usepackage[margin=1in]{geometry}
\usepackage{color}
\usepackage{fancyvrb}
\newcommand{\VerbBar}{|}
\newcommand{\VERB}{\Verb[commandchars=\\\{\}]}
\DefineVerbatimEnvironment{Highlighting}{Verbatim}{commandchars=\\\{\}}
% Add ',fontsize=\small' for more characters per line
\usepackage{framed}
\definecolor{shadecolor}{RGB}{248,248,248}
\newenvironment{Shaded}{\begin{snugshade}}{\end{snugshade}}
\newcommand{\AlertTok}[1]{\textcolor[rgb]{0.94,0.16,0.16}{#1}}
\newcommand{\AnnotationTok}[1]{\textcolor[rgb]{0.56,0.35,0.01}{\textbf{\textit{#1}}}}
\newcommand{\AttributeTok}[1]{\textcolor[rgb]{0.77,0.63,0.00}{#1}}
\newcommand{\BaseNTok}[1]{\textcolor[rgb]{0.00,0.00,0.81}{#1}}
\newcommand{\BuiltInTok}[1]{#1}
\newcommand{\CharTok}[1]{\textcolor[rgb]{0.31,0.60,0.02}{#1}}
\newcommand{\CommentTok}[1]{\textcolor[rgb]{0.56,0.35,0.01}{\textit{#1}}}
\newcommand{\CommentVarTok}[1]{\textcolor[rgb]{0.56,0.35,0.01}{\textbf{\textit{#1}}}}
\newcommand{\ConstantTok}[1]{\textcolor[rgb]{0.00,0.00,0.00}{#1}}
\newcommand{\ControlFlowTok}[1]{\textcolor[rgb]{0.13,0.29,0.53}{\textbf{#1}}}
\newcommand{\DataTypeTok}[1]{\textcolor[rgb]{0.13,0.29,0.53}{#1}}
\newcommand{\DecValTok}[1]{\textcolor[rgb]{0.00,0.00,0.81}{#1}}
\newcommand{\DocumentationTok}[1]{\textcolor[rgb]{0.56,0.35,0.01}{\textbf{\textit{#1}}}}
\newcommand{\ErrorTok}[1]{\textcolor[rgb]{0.64,0.00,0.00}{\textbf{#1}}}
\newcommand{\ExtensionTok}[1]{#1}
\newcommand{\FloatTok}[1]{\textcolor[rgb]{0.00,0.00,0.81}{#1}}
\newcommand{\FunctionTok}[1]{\textcolor[rgb]{0.00,0.00,0.00}{#1}}
\newcommand{\ImportTok}[1]{#1}
\newcommand{\InformationTok}[1]{\textcolor[rgb]{0.56,0.35,0.01}{\textbf{\textit{#1}}}}
\newcommand{\KeywordTok}[1]{\textcolor[rgb]{0.13,0.29,0.53}{\textbf{#1}}}
\newcommand{\NormalTok}[1]{#1}
\newcommand{\OperatorTok}[1]{\textcolor[rgb]{0.81,0.36,0.00}{\textbf{#1}}}
\newcommand{\OtherTok}[1]{\textcolor[rgb]{0.56,0.35,0.01}{#1}}
\newcommand{\PreprocessorTok}[1]{\textcolor[rgb]{0.56,0.35,0.01}{\textit{#1}}}
\newcommand{\RegionMarkerTok}[1]{#1}
\newcommand{\SpecialCharTok}[1]{\textcolor[rgb]{0.00,0.00,0.00}{#1}}
\newcommand{\SpecialStringTok}[1]{\textcolor[rgb]{0.31,0.60,0.02}{#1}}
\newcommand{\StringTok}[1]{\textcolor[rgb]{0.31,0.60,0.02}{#1}}
\newcommand{\VariableTok}[1]{\textcolor[rgb]{0.00,0.00,0.00}{#1}}
\newcommand{\VerbatimStringTok}[1]{\textcolor[rgb]{0.31,0.60,0.02}{#1}}
\newcommand{\WarningTok}[1]{\textcolor[rgb]{0.56,0.35,0.01}{\textbf{\textit{#1}}}}
\usepackage{graphicx}
\makeatletter
\def\maxwidth{\ifdim\Gin@nat@width>\linewidth\linewidth\else\Gin@nat@width\fi}
\def\maxheight{\ifdim\Gin@nat@height>\textheight\textheight\else\Gin@nat@height\fi}
\makeatother
% Scale images if necessary, so that they will not overflow the page
% margins by default, and it is still possible to overwrite the defaults
% using explicit options in \includegraphics[width, height, ...]{}
\setkeys{Gin}{width=\maxwidth,height=\maxheight,keepaspectratio}
% Set default figure placement to htbp
\makeatletter
\def\fps@figure{htbp}
\makeatother
\setlength{\emergencystretch}{3em} % prevent overfull lines
\providecommand{\tightlist}{%
  \setlength{\itemsep}{0pt}\setlength{\parskip}{0pt}}
\setcounter{secnumdepth}{-\maxdimen} % remove section numbering
\ifluatex
  \usepackage{selnolig}  % disable illegal ligatures
\fi

\title{Chapter 4-(1)}
\author{}
\date{\vspace{-2.5em}}

\begin{document}
\maketitle

\hypertarget{chapter-4-the-tidyverse}{%
\section{Chapter 4 The tidyverse}\label{chapter-4-the-tidyverse}}

Up to now we have been manipulating vectors by reordering and subsetting
them through indexing. However, once we start more advanced analyses,
the preferred unit for data storage is not the vector but the data
frame. In this chapter we learn to work directly with data frames, which
greatly facilitate the organization of information. We will be using
data frames for the majority of this book. We will focus on a specific
data format referred to as tidy and on specific collection of packages
that are particularly helpful for working with tidy data referred to as
the tidyverse.

We can load all the tidyverse packages at once by installing and loading
the tidyverse package:

\begin{Shaded}
\begin{Highlighting}[]
\FunctionTok{library}\NormalTok{(tidyverse)}
\end{Highlighting}
\end{Shaded}

\begin{verbatim}
## -- Attaching packages --------------------------------------- tidyverse 1.3.1 --
\end{verbatim}

\begin{verbatim}
## v ggplot2 3.3.5     v purrr   0.3.4
## v tibble  3.1.3     v dplyr   1.0.7
## v tidyr   1.1.3     v stringr 1.4.0
## v readr   2.0.1     v forcats 0.5.1
\end{verbatim}

\begin{verbatim}
## -- Conflicts ------------------------------------------ tidyverse_conflicts() --
## x dplyr::filter() masks stats::filter()
## x dplyr::lag()    masks stats::lag()
\end{verbatim}

We will learn how to implement the tidyverse approach throughout the
book, but before delving into the details, in this chapter we introduce
some of the most widely used tidyverse functionality, starting with the
dplyr package for manipulating data frames and the purrr package for
working with functions. Note that the tidyverse also includes a graphing
package, ggplot2, which we introduce later in Chapter 7 in the Data
Visualization part of the book; the readr package discussed in Chapter
5; and many others. In this chapter, we first introduce the concept of
tidy data and then demonstrate how we use the tidyverse to work with
data frames in this format.

\hypertarget{tidy-data}{%
\subsection{4.1 Tidy data}\label{tidy-data}}

We say that a data table is in tidy format if each row represents one
observation and columns represent the different variables available for
each of these observations. The murders dataset is an example of a tidy
data frame.

\begin{Shaded}
\begin{Highlighting}[]
\FunctionTok{library}\NormalTok{(dslabs)}
\FunctionTok{data}\NormalTok{(murders)}
\FunctionTok{head}\NormalTok{(murders)}
\end{Highlighting}
\end{Shaded}

\begin{verbatim}
##        state abb region population total
## 1    Alabama  AL  South    4779736   135
## 2     Alaska  AK   West     710231    19
## 3    Arizona  AZ   West    6392017   232
## 4   Arkansas  AR  South    2915918    93
## 5 California  CA   West   37253956  1257
## 6   Colorado  CO   West    5029196    65
\end{verbatim}

Each row represent a state with each of the five columns providing a
different variable related to these states: name, abbreviation, region,
population, and total murders.

To see how the same information can be provided in different formats,
consider the following example:

This tidy dataset provides fertility rates for two countries across the
years. This is a tidy dataset because each row presents one observation
with the three variables being country, year, and fertility rate.
However, this dataset originally came in another format and was reshaped
for the dslabs package. Originally, the data was in the following
format:

The same information is provided, but there are two important
differences in the format: 1) each row includes several observations and
2) one of the variables, year, is stored in the header. For the
tidyverse packages to be optimally used, data need to be reshaped into
tidy format, which you will learn to do in the Data Wrangling part of
the book. Until then, we will use example datasets that are already in
tidy format.

Although not immediately obvious, as you go through the book you will
start to appreciate the advantages of working in a framework in which
functions use tidy formats for both inputs and outputs. You will see how
this permits the data analyst to focus on more important aspects of the
analysis rather than the format of the data.

\hypertarget{exercises}{%
\subsection{4.2 Exercises}\label{exercises}}

\begin{enumerate}
\def\labelenumi{\arabic{enumi}.}
\tightlist
\item
  Examine the built-in datset co2. Which of the following is true:
\end{enumerate}

\begin{enumerate}
\def\labelenumi{\alph{enumi}.}
\tightlist
\item
  co2 is tidy data: it has one year for each row.
\item
  co2 is not tidy: we need at least one column with a character vector.
\item
  co2 is not tidy: it is a matrix instead of a data frame.
\item
  co2 is not tidy: to be tidy we would have to wrangle it to have three
  columns(year, month and value), then each co2 observation would have a
  row.
\end{enumerate}

\#d

\begin{enumerate}
\def\labelenumi{\arabic{enumi}.}
\setcounter{enumi}{1}
\tightlist
\item
  Examine the built-in dataset ChickWeight. Which of the following is
  true:
\end{enumerate}

\begin{enumerate}
\def\labelenumi{\alph{enumi}.}
\tightlist
\item
  ChickWeight is not tidy: each chick has more than one row.
\item
  ChickWeight is tidy: each observation (a weight) is represented by one
  row. The chick from which this measurement came is one of the
  variables.
\item
  ChickWeight is not tidy: we are missing the year column.
\item
  ChickWeight is tidy: it is stored in a data frame.
\end{enumerate}

\#d

\begin{enumerate}
\def\labelenumi{\arabic{enumi}.}
\setcounter{enumi}{2}
\tightlist
\item
  Examine the built-in dataset BOD. Which of the following is true:
\end{enumerate}

\begin{enumerate}
\def\labelenumi{\alph{enumi}.}
\tightlist
\item
  BOD is not tidy: it only has six rows.
\item
  BOD is not tidy: the first column is just an index.
\item
  BOD is tidy: each row is an observation with two values (time and
  demand)
\item
  BOD is tidy: all small datasets are tidy by definition.
\end{enumerate}

\#c

\begin{enumerate}
\def\labelenumi{\arabic{enumi}.}
\setcounter{enumi}{3}
\tightlist
\item
  Which of the following built-in datasets is tidy (you can pick more
  than one):
\end{enumerate}

\begin{enumerate}
\def\labelenumi{\alph{enumi}.}
\tightlist
\item
  BJsales
\item
  EuStockMarkets
\item
  DNase
\item
  Formaldehyde
\item
  Orange
\item
  UCBAdmissions
\end{enumerate}

\#c,d,e

\hypertarget{manipulating-data-frames}{%
\subsection{4.3 Manipulating data
frames}\label{manipulating-data-frames}}

The dplyr package from the tidyverse introduces functions that perform
some of the most common operations when working with data frames and
uses names for these functions that are relatively easy to remember. For
instance, to change the data table by adding a new column, we use
mutate. To filter the data table to a subset of rows, we use filter.
Finally, to subset the data by selecting specific columns, we use
select.

\hypertarget{adding-a-column-with-mutate}{%
\subsubsection{4.3.1 Adding a column with
mutate}\label{adding-a-column-with-mutate}}

We want all the necessary information for our analysis to be included in
the data table. So the first task is to add the murder rates to our
murders data frame. The function mutate takes the data frame as a first
argument and the name and values of the variable as a second argument
using the convention name=values. So, to add murder rates, we use:

\begin{Shaded}
\begin{Highlighting}[]
\FunctionTok{library}\NormalTok{(dslabs)}
\FunctionTok{data}\NormalTok{(murders)}
\NormalTok{murders}\OtherTok{\textless{}{-}}\FunctionTok{mutate}\NormalTok{(murders, }\AttributeTok{rate=}\NormalTok{ total}\SpecialCharTok{/}\NormalTok{population}\SpecialCharTok{*}\DecValTok{100000}\NormalTok{)}
\end{Highlighting}
\end{Shaded}

Notice that here we used total and population inside the function, which
are objects that are not defined in our workspace. But why don't we get
an error?

This is one of dplyr's main features. Functions in this packages, such
as mutate, know to look for vaiables in the data frame provided in the
first argument. In the call to mutate above, total will have the values
in murders\$total. This approach makes the code much more readable.

We can see that the new column is added:

\begin{Shaded}
\begin{Highlighting}[]
\FunctionTok{head}\NormalTok{(murders)}
\end{Highlighting}
\end{Shaded}

\begin{verbatim}
##        state abb region population total     rate
## 1    Alabama  AL  South    4779736   135 2.824424
## 2     Alaska  AK   West     710231    19 2.675186
## 3    Arizona  AZ   West    6392017   232 3.629527
## 4   Arkansas  AR  South    2915918    93 3.189390
## 5 California  CA   West   37253956  1257 3.374138
## 6   Colorado  CO   West    5029196    65 1.292453
\end{verbatim}

Although we have overwritten the original murders object, this does not
change the object that loaded with data(murders). If we load the murders
data again, the original will overwrite our mutated version.

\hypertarget{subsetting-with-filter}{%
\subsubsection{4.3.2 Subsetting with
filter}\label{subsetting-with-filter}}

Now suppose that we want to filter the data table to only show the
entries for which the murder rate is lower than 0.71. To do this we use
the filter function, which takes the data table as the first argument
and then the conditional statement as the second. Like mutate, we can
use the unquoated variable names from murders inside the function and it
will know we mean the columns and not objects in the workspace.

\begin{Shaded}
\begin{Highlighting}[]
\FunctionTok{filter}\NormalTok{(murders, rate}\SpecialCharTok{\textless{}=}\FloatTok{0.71}\NormalTok{)}
\end{Highlighting}
\end{Shaded}

\begin{verbatim}
##           state abb        region population total      rate
## 1        Hawaii  HI          West    1360301     7 0.5145920
## 2          Iowa  IA North Central    3046355    21 0.6893484
## 3 New Hampshire  NH     Northeast    1316470     5 0.3798036
## 4  North Dakota  ND North Central     672591     4 0.5947151
## 5       Vermont  VT     Northeast     625741     2 0.3196211
\end{verbatim}

\hypertarget{selecting-columns-with-select}{%
\subsubsection{4.3.3 Selecting columns with
select}\label{selecting-columns-with-select}}

Although our data table only has six columns, some data tables include
hundreds. If we want to view just a few, we can use the dplyr select
function. In the code below we select three columns, assign this to a
new object and then filter the new object:

\begin{Shaded}
\begin{Highlighting}[]
\NormalTok{new\_table}\OtherTok{\textless{}{-}} \FunctionTok{select}\NormalTok{(murders, state, region, rate)}
\FunctionTok{filter}\NormalTok{(new\_table, rate}\SpecialCharTok{\textless{}=} \FloatTok{0.71}\NormalTok{)}
\end{Highlighting}
\end{Shaded}

\begin{verbatim}
##           state        region      rate
## 1        Hawaii          West 0.5145920
## 2          Iowa North Central 0.6893484
## 3 New Hampshire     Northeast 0.3798036
## 4  North Dakota North Central 0.5947151
## 5       Vermont     Northeast 0.3196211
\end{verbatim}

In the call to select, the first argument murders is an object, but
state, region, and rate are variable names.

\hypertarget{exercises-1}{%
\subsection{4.4 Exercises}\label{exercises-1}}

\begin{enumerate}
\def\labelenumi{\arabic{enumi}.}
\tightlist
\item
  Load the dplyr package and the murders dataset.
\end{enumerate}

\begin{Shaded}
\begin{Highlighting}[]
\FunctionTok{library}\NormalTok{(dplyr)}
\FunctionTok{library}\NormalTok{(dslabs)}
\FunctionTok{data}\NormalTok{(murders)}
\end{Highlighting}
\end{Shaded}

You can add columns using the dplyr function mutate. This function is
aware of the column names and inside the function you can call them
unquoted:

\begin{Shaded}
\begin{Highlighting}[]
\NormalTok{murders}\OtherTok{\textless{}{-}} \FunctionTok{mutate}\NormalTok{(murders, }\AttributeTok{population\_in\_millions=}\NormalTok{population}\SpecialCharTok{/}\DecValTok{10}\SpecialCharTok{\^{}}\DecValTok{6}\NormalTok{)}
\end{Highlighting}
\end{Shaded}

We can write population rather than murders\$population. The function
mutate knows we are grabbing columns from murders. Use the function
mutate to add a murders column named rate with the per 100,000 murder
rate as in the example code above. Make sure you redefine murders as
done in the example code above(murders\textless- {[}your code{]}) so we
can keep using this variable.

\begin{Shaded}
\begin{Highlighting}[]
\FunctionTok{library}\NormalTok{(dslabs)}
\FunctionTok{data}\NormalTok{(murders)}
\NormalTok{murders }\OtherTok{\textless{}{-}} \FunctionTok{mutate}\NormalTok{(murders,}\AttributeTok{rate=}\NormalTok{total}\SpecialCharTok{/}\NormalTok{population}\SpecialCharTok{*}\DecValTok{100000}\NormalTok{ )}
\end{Highlighting}
\end{Shaded}

\begin{enumerate}
\def\labelenumi{\arabic{enumi}.}
\setcounter{enumi}{1}
\tightlist
\item
  If rank(x) gives you the ranks of x from lowest to highest, rank(-x)
  gives you the ranks from highest to lowest. Use the function mutate to
  add a column rank containing the rank, from highest to lowest murder
  rate. Make sure you redefine murders so we can keep using this
  variable.
\end{enumerate}

\begin{Shaded}
\begin{Highlighting}[]
\NormalTok{murders}\OtherTok{\textless{}{-}}\FunctionTok{mutate}\NormalTok{(murders, }\AttributeTok{rank=}\FunctionTok{rank}\NormalTok{(}\SpecialCharTok{{-}}\NormalTok{rate))}
\end{Highlighting}
\end{Shaded}

\begin{enumerate}
\def\labelenumi{\arabic{enumi}.}
\setcounter{enumi}{2}
\tightlist
\item
  With dplyr, we can use select to show only certain columns. For
  example, with this code we would only show the states and population
  sizes:
\end{enumerate}

\begin{Shaded}
\begin{Highlighting}[]
\FunctionTok{select}\NormalTok{(murders, state, population) }\SpecialCharTok{\%\textgreater{}\%} \FunctionTok{head}\NormalTok{()}
\end{Highlighting}
\end{Shaded}

\begin{verbatim}
##        state population
## 1    Alabama    4779736
## 2     Alaska     710231
## 3    Arizona    6392017
## 4   Arkansas    2915918
## 5 California   37253956
## 6   Colorado    5029196
\end{verbatim}

Use select to show the state names and abbreviations in murders. Do not
redefine murders, just show the results.

\begin{Shaded}
\begin{Highlighting}[]
\FunctionTok{select}\NormalTok{(murders, state, abb) }\SpecialCharTok{\%\textgreater{}\%} \FunctionTok{head}\NormalTok{()}
\end{Highlighting}
\end{Shaded}

\begin{verbatim}
##        state abb
## 1    Alabama  AL
## 2     Alaska  AK
## 3    Arizona  AZ
## 4   Arkansas  AR
## 5 California  CA
## 6   Colorado  CO
\end{verbatim}

\begin{enumerate}
\def\labelenumi{\arabic{enumi}.}
\setcounter{enumi}{3}
\tightlist
\item
  The dplyr function filter is used to choose specific rows of the data
  frame to keep. Unlike select which is for columns, filter is for rows.
  For example, you can show just the New York row like this:
\end{enumerate}

\begin{Shaded}
\begin{Highlighting}[]
\FunctionTok{filter}\NormalTok{(murders, state}\SpecialCharTok{==}\StringTok{"New York"}\NormalTok{)}
\end{Highlighting}
\end{Shaded}

\begin{verbatim}
##      state abb    region population total    rate rank
## 1 New York  NY Northeast   19378102   517 2.66796   29
\end{verbatim}

You can use other logical vectors to filter rows.

Use filter to show the top 5 states with the highest murder rates. After
we add murder rate and rank, do not change the murders dataset, just
show the result. Remember that you can filter based on the rank column.

\begin{Shaded}
\begin{Highlighting}[]
\FunctionTok{filter}\NormalTok{(murders, rank}\SpecialCharTok{\textless{}=}\DecValTok{5}\NormalTok{) }
\end{Highlighting}
\end{Shaded}

\begin{verbatim}
##                  state abb        region population total      rate rank
## 1 District of Columbia  DC         South     601723    99 16.452753    1
## 2            Louisiana  LA         South    4533372   351  7.742581    2
## 3             Maryland  MD         South    5773552   293  5.074866    4
## 4             Missouri  MO North Central    5988927   321  5.359892    3
## 5       South Carolina  SC         South    4625364   207  4.475323    5
\end{verbatim}

\begin{enumerate}
\def\labelenumi{\arabic{enumi}.}
\setcounter{enumi}{4}
\tightlist
\item
  We can remove rows using the != operator. For example, to remove
  Floida, we would do this:
\end{enumerate}

\begin{Shaded}
\begin{Highlighting}[]
\NormalTok{no\_florida}\OtherTok{\textless{}{-}} \FunctionTok{filter}\NormalTok{(murders, state }\SpecialCharTok{!=} \StringTok{"Florida"}\NormalTok{)}
\end{Highlighting}
\end{Shaded}

Create a new data frame called no\_south that removes states from the
South region. How many states are in this category. You can use the
function nrow for this.

\begin{Shaded}
\begin{Highlighting}[]
\NormalTok{no\_south}\OtherTok{\textless{}{-}} \FunctionTok{filter}\NormalTok{(murders, region }\SpecialCharTok{!=} \StringTok{"South"}\NormalTok{)}
\FunctionTok{nrow}\NormalTok{(no\_south)}
\end{Highlighting}
\end{Shaded}

\begin{verbatim}
## [1] 34
\end{verbatim}

\begin{enumerate}
\def\labelenumi{\arabic{enumi}.}
\setcounter{enumi}{5}
\tightlist
\item
  We can also use \%in\% to filter with dplyr. You can therefore see the
  data from New York and Texas like this:
\end{enumerate}

\begin{Shaded}
\begin{Highlighting}[]
\FunctionTok{filter}\NormalTok{(murders, state }\SpecialCharTok{\%in\%} \FunctionTok{c}\NormalTok{(}\StringTok{"New York"}\NormalTok{, }\StringTok{"Texas"}\NormalTok{))}
\end{Highlighting}
\end{Shaded}

\begin{verbatim}
##      state abb    region population total    rate rank
## 1 New York  NY Northeast   19378102   517 2.66796   29
## 2    Texas  TX     South   25145561   805 3.20136   16
\end{verbatim}

Create a new data frame called murders\_nw with only the states from the
Northeast and the West. How many states are in this category?

\begin{Shaded}
\begin{Highlighting}[]
\NormalTok{murders\_nw}\OtherTok{\textless{}{-}}\FunctionTok{filter}\NormalTok{(murders, region }\SpecialCharTok{\%in\%} \FunctionTok{c}\NormalTok{(}\StringTok{"Northeast"}\NormalTok{, }\StringTok{"West"}\NormalTok{))}
\FunctionTok{nrow}\NormalTok{(murders\_nw)}
\end{Highlighting}
\end{Shaded}

\begin{verbatim}
## [1] 22
\end{verbatim}

\begin{enumerate}
\def\labelenumi{\arabic{enumi}.}
\setcounter{enumi}{6}
\tightlist
\item
  Suppose you want to live in the Northeast or West and want the murder
  rate to be less than 1. We want to see the data for the states
  satisfying these options. Note that you can use logical operators with
  filter. Here is an example in which we filter to keep only small
  states in the Northeast region.
\end{enumerate}

\begin{Shaded}
\begin{Highlighting}[]
\FunctionTok{filter}\NormalTok{(murders, population }\SpecialCharTok{\textless{}} \DecValTok{5000000} \SpecialCharTok{\&}\NormalTok{ region }\SpecialCharTok{==}\StringTok{"Northeast"}\NormalTok{)}
\end{Highlighting}
\end{Shaded}

\begin{verbatim}
##           state abb    region population total      rate rank
## 1   Connecticut  CT Northeast    3574097    97 2.7139722   25
## 2         Maine  ME Northeast    1328361    11 0.8280881   44
## 3 New Hampshire  NH Northeast    1316470     5 0.3798036   50
## 4  Rhode Island  RI Northeast    1052567    16 1.5200933   35
## 5       Vermont  VT Northeast     625741     2 0.3196211   51
\end{verbatim}

Make sure murders has been defined with rate and rank and still has all
states. Create a table callsed my\_states that contain rows for states
satisfying both the conditions: it is in the Northeast or West and the
murder rate is less than 1. Use select to show only the state name, the
rate, and the rank.

\begin{Shaded}
\begin{Highlighting}[]
\NormalTok{my\_states}\OtherTok{\textless{}{-}}\FunctionTok{filter}\NormalTok{(murders, region }\SpecialCharTok{\%in\%} \FunctionTok{c}\NormalTok{(}\StringTok{"Northeast"}\NormalTok{,}\StringTok{"West"}\NormalTok{) }\SpecialCharTok{\&}\NormalTok{ rate }\SpecialCharTok{\textless{}}\DecValTok{1}\NormalTok{)}
\FunctionTok{select}\NormalTok{(my\_states, state, rate, rank)}
\end{Highlighting}
\end{Shaded}

\begin{verbatim}
##           state      rate rank
## 1        Hawaii 0.5145920   49
## 2         Idaho 0.7655102   46
## 3         Maine 0.8280881   44
## 4 New Hampshire 0.3798036   50
## 5        Oregon 0.9396843   42
## 6          Utah 0.7959810   45
## 7       Vermont 0.3196211   51
## 8       Wyoming 0.8871131   43
\end{verbatim}

\hypertarget{the-pipe}{%
\section{4.5 The pipe: \%\textgreater\%}\label{the-pipe}}

With dplyr we can perform a series of operations, for example select and
then filter, by sending the results of one function to another using
what is called the pipe operator: \%\textgreater\% . Some details are
included below.

We wrote code above to show three variables (state, region, rate) for
states that have murder rates below 0.71. To do this, we defined the
intermediate object new\_table. In dplyr we can write code that looks
more like a description of what we want to do without intermediate
objects:

original data -\textgreater{} select -\textgreater{} filter

For such an operation, we can use the pipe \%\textgreater\% . The code
looks like this:

\begin{Shaded}
\begin{Highlighting}[]
\NormalTok{murders }\SpecialCharTok{\%\textgreater{}\%}  \FunctionTok{select}\NormalTok{(state, region, rate) }\SpecialCharTok{\%\textgreater{}\%}  \FunctionTok{filter}\NormalTok{(rate}\SpecialCharTok{\textless{}=}\FloatTok{0.71}\NormalTok{)}
\end{Highlighting}
\end{Shaded}

\begin{verbatim}
##           state        region      rate
## 1        Hawaii          West 0.5145920
## 2          Iowa North Central 0.6893484
## 3 New Hampshire     Northeast 0.3798036
## 4  North Dakota North Central 0.5947151
## 5       Vermont     Northeast 0.3196211
\end{verbatim}

This line of code is equivalent to the two lines of code above. What is
going on here?

In general, the pipe sends the result of the left side of the pipe to be
the first argument of the function on the right side of the pipe. Here
is a very simple example:

\begin{Shaded}
\begin{Highlighting}[]
\DecValTok{16} \SpecialCharTok{\%\textgreater{}\%} \FunctionTok{sqrt}\NormalTok{()}
\end{Highlighting}
\end{Shaded}

\begin{verbatim}
## [1] 4
\end{verbatim}

We can continue to pipe values along:

\begin{Shaded}
\begin{Highlighting}[]
\DecValTok{16} \SpecialCharTok{\%\textgreater{}\%} \FunctionTok{sqrt}\NormalTok{() }\SpecialCharTok{\%\textgreater{}\%} \FunctionTok{log2}\NormalTok{()}
\end{Highlighting}
\end{Shaded}

\begin{verbatim}
## [1] 2
\end{verbatim}

The above statement is equivalent to log2(sqrt(16)).

Remember that the pipe sends values to the first argument, so we can
define other arguments as if the first argument is already defined:

\begin{Shaded}
\begin{Highlighting}[]
\DecValTok{16} \SpecialCharTok{\%\textgreater{}\%} \FunctionTok{sqrt}\NormalTok{() }\SpecialCharTok{\%\textgreater{}\%} \FunctionTok{log}\NormalTok{(}\AttributeTok{base=}\DecValTok{2}\NormalTok{)}
\end{Highlighting}
\end{Shaded}

\begin{verbatim}
## [1] 2
\end{verbatim}

Therefore, when using the pipe with data frames and dplyr, we no longer
need to specify the required first argument since the dplyr functions we
have described all take the data as the first argument. In the code we
wrote:

\begin{Shaded}
\begin{Highlighting}[]
\NormalTok{murders }\SpecialCharTok{\%\textgreater{}\%} \FunctionTok{select}\NormalTok{(state, region, rate) }\SpecialCharTok{\%\textgreater{}\%} \FunctionTok{filter}\NormalTok{(rate }\SpecialCharTok{\textless{}=} \FloatTok{0.71}\NormalTok{)}
\end{Highlighting}
\end{Shaded}

\begin{verbatim}
##           state        region      rate
## 1        Hawaii          West 0.5145920
## 2          Iowa North Central 0.6893484
## 3 New Hampshire     Northeast 0.3798036
## 4  North Dakota North Central 0.5947151
## 5       Vermont     Northeast 0.3196211
\end{verbatim}

murders is the first argument of the select function, and the new data
frame(formerly new\_table) is the first argument of the filter function.
Note that the pipe works well with functions where the first argument is
the input data. Functions in tidyverse packages like dplyr have this
format and can be used easily with the pipe.

\hypertarget{exercises-2}{%
\subsection{4.6 Exercises}\label{exercises-2}}

\begin{enumerate}
\def\labelenumi{\arabic{enumi}.}
\tightlist
\item
  The pipe \%\textgreater\% can be used to perform operations
  sequentially without having to define intermediate objects. Start by
  redefining murder to include rate and rank.
\end{enumerate}

\begin{Shaded}
\begin{Highlighting}[]
\NormalTok{murders }\OtherTok{\textless{}{-}} \FunctionTok{mutate}\NormalTok{(murders, }\AttributeTok{rate =}\NormalTok{  total }\SpecialCharTok{/}\NormalTok{ population }\SpecialCharTok{*} \DecValTok{100000}\NormalTok{, }\AttributeTok{rank =} \FunctionTok{rank}\NormalTok{(}\SpecialCharTok{{-}}\NormalTok{rate))}
\end{Highlighting}
\end{Shaded}

In the solution to the previous exercise, we did the following:

\begin{Shaded}
\begin{Highlighting}[]
\NormalTok{my\_states }\OtherTok{\textless{}{-}} \FunctionTok{filter}\NormalTok{(murders, region }\SpecialCharTok{\%in\%} \FunctionTok{c}\NormalTok{(}\StringTok{"Northeast"}\NormalTok{, }\StringTok{"West"}\NormalTok{) }\SpecialCharTok{\&}\NormalTok{ rate }\SpecialCharTok{\textless{}}\DecValTok{1}\NormalTok{)}
\FunctionTok{select}\NormalTok{(my\_states, state, rate, rank)}
\end{Highlighting}
\end{Shaded}

\begin{verbatim}
##           state      rate rank
## 1        Hawaii 0.5145920   49
## 2         Idaho 0.7655102   46
## 3         Maine 0.8280881   44
## 4 New Hampshire 0.3798036   50
## 5        Oregon 0.9396843   42
## 6          Utah 0.7959810   45
## 7       Vermont 0.3196211   51
## 8       Wyoming 0.8871131   43
\end{verbatim}

The pipe \%\textgreater\% permits us to perform both operations
sequentially without having to define an intermediate variable
my\_states. We therefore could have mutated and selected in the same
line like this:

\begin{Shaded}
\begin{Highlighting}[]
\FunctionTok{mutate}\NormalTok{(murders, }\AttributeTok{rate =}\NormalTok{  total }\SpecialCharTok{/}\NormalTok{ population }\SpecialCharTok{*} \DecValTok{100000}\NormalTok{, }\AttributeTok{rank =} \FunctionTok{rank}\NormalTok{(}\SpecialCharTok{{-}}\NormalTok{rate)) }\SpecialCharTok{\%\textgreater{}\%}
  \FunctionTok{select}\NormalTok{(state, rate, rank)}
\end{Highlighting}
\end{Shaded}

\begin{verbatim}
##                   state       rate rank
## 1               Alabama  2.8244238   23
## 2                Alaska  2.6751860   27
## 3               Arizona  3.6295273   10
## 4              Arkansas  3.1893901   17
## 5            California  3.3741383   14
## 6              Colorado  1.2924531   38
## 7           Connecticut  2.7139722   25
## 8              Delaware  4.2319369    6
## 9  District of Columbia 16.4527532    1
## 10              Florida  3.3980688   13
## 11              Georgia  3.7903226    9
## 12               Hawaii  0.5145920   49
## 13                Idaho  0.7655102   46
## 14             Illinois  2.8369608   22
## 15              Indiana  2.1900730   31
## 16                 Iowa  0.6893484   47
## 17               Kansas  2.2081106   30
## 18             Kentucky  2.6732010   28
## 19            Louisiana  7.7425810    2
## 20                Maine  0.8280881   44
## 21             Maryland  5.0748655    4
## 22        Massachusetts  1.8021791   32
## 23             Michigan  4.1786225    7
## 24            Minnesota  0.9992600   40
## 25          Mississippi  4.0440846    8
## 26             Missouri  5.3598917    3
## 27              Montana  1.2128379   39
## 28             Nebraska  1.7521372   33
## 29               Nevada  3.1104763   19
## 30        New Hampshire  0.3798036   50
## 31           New Jersey  2.7980319   24
## 32           New Mexico  3.2537239   15
## 33             New York  2.6679599   29
## 34       North Carolina  2.9993237   20
## 35         North Dakota  0.5947151   48
## 36                 Ohio  2.6871225   26
## 37             Oklahoma  2.9589340   21
## 38               Oregon  0.9396843   42
## 39         Pennsylvania  3.5977513   11
## 40         Rhode Island  1.5200933   35
## 41       South Carolina  4.4753235    5
## 42         South Dakota  0.9825837   41
## 43            Tennessee  3.4509357   12
## 44                Texas  3.2013603   16
## 45                 Utah  0.7959810   45
## 46              Vermont  0.3196211   51
## 47             Virginia  3.1246001   18
## 48           Washington  1.3829942   37
## 49        West Virginia  1.4571013   36
## 50            Wisconsin  1.7056487   34
## 51              Wyoming  0.8871131   43
\end{verbatim}

Notice the select no longer has a data frame as the first argument. They
first argument is assumed to be the result of the operation conducted
right before the \%\textgreater\%.

Repeat the previous exercise, but now instead of creating a new object,
show the result and only include the state, rate, and rank columns. Use
a pipe \%\textgreater\% to do this in just one line.

\begin{Shaded}
\begin{Highlighting}[]
\NormalTok{new\_object}\OtherTok{\textless{}{-}}\FunctionTok{filter}\NormalTok{(murders, region }\SpecialCharTok{\%in\%} \FunctionTok{c}\NormalTok{(}\StringTok{"Northeast"}\NormalTok{,}\StringTok{"West"}\NormalTok{) }\SpecialCharTok{\&}\NormalTok{ rate }\SpecialCharTok{\textless{}}\DecValTok{1}\NormalTok{) }\SpecialCharTok{\%\textgreater{}\%} 
\FunctionTok{select}\NormalTok{(state, rate, rank)}
\end{Highlighting}
\end{Shaded}

\begin{enumerate}
\def\labelenumi{\arabic{enumi}.}
\setcounter{enumi}{1}
\tightlist
\item
  Reset murders to the original table by using data(murders). Use a pipe
  to create a new data frame called my\_states that considers only
  states in the Northeast or West which have a murder rate lower than 1,
  and contains only the state, rate and rank columns. The pipe should
  also have four components separated by three \%\textgreater\%. The
  code should look something like this:
\end{enumerate}

\begin{Shaded}
\begin{Highlighting}[]
\CommentTok{\#my\_states \textless{}{-} murders \%\textgreater{}\%}
  \CommentTok{\#mutate SOMETHING \%\textgreater{}\% }
  \CommentTok{\#filter SOMETHING \%\textgreater{}\% }
  \CommentTok{\#select SOMETHING}
\end{Highlighting}
\end{Shaded}

\begin{Shaded}
\begin{Highlighting}[]
\FunctionTok{data}\NormalTok{(murders)}
\NormalTok{my\_states}\OtherTok{\textless{}{-}}\NormalTok{murders }\SpecialCharTok{\%\textgreater{}\%} 
  \FunctionTok{mutate}\NormalTok{(}\AttributeTok{rate =}\NormalTok{  total }\SpecialCharTok{/}\NormalTok{ population }\SpecialCharTok{*} \DecValTok{100000}\NormalTok{, }\AttributeTok{rank =} \FunctionTok{rank}\NormalTok{(}\SpecialCharTok{{-}}\NormalTok{rate)) }\SpecialCharTok{\%\textgreater{}\%} 
  \FunctionTok{filter}\NormalTok{(region }\SpecialCharTok{\%in\%} \FunctionTok{c}\NormalTok{(}\StringTok{"Northeast"}\NormalTok{,}\StringTok{"West"}\NormalTok{) }\SpecialCharTok{\&}\NormalTok{ rate }\SpecialCharTok{\textless{}}\DecValTok{1}\NormalTok{) }\SpecialCharTok{\%\textgreater{}\%} 
  \FunctionTok{select}\NormalTok{(state, rate, rank)}
\NormalTok{my\_states}
\end{Highlighting}
\end{Shaded}

\begin{verbatim}
##           state      rate rank
## 1        Hawaii 0.5145920   49
## 2         Idaho 0.7655102   46
## 3         Maine 0.8280881   44
## 4 New Hampshire 0.3798036   50
## 5        Oregon 0.9396843   42
## 6          Utah 0.7959810   45
## 7       Vermont 0.3196211   51
## 8       Wyoming 0.8871131   43
\end{verbatim}

\end{document}
